\section{System Overview}
In  order to analyze the performance effects of the database cracking, we have implemented our own simple database system MiniDB. In this section we will describe components and implementation details of MiniDB.

MiniDB is a single column database implemented in Java, where each table consists of a single uniquely named column. The database keeps a mapping between a column name and a  column object in a hash map. The database table stores tuples as a list, and each tuple is a 32-bit integer. MiniDB maintains the following data structures: Simple Column, Sorted Column, Cracker Column, Cracker Index, Range Scan. 

\label{sec:system}

\subsection{MiniDB columns and indexes}


Simple Columns represent database tables and store tuples. They do not maintain a specific tuple order and insert a tuple to the end of the tuple list. In Simple Columns tuple lookup requires linear scan of the tuples list. 

Sorted Column is another data structure that represents database tables  which preserves the order of the tuples. Tuple lookup takes logarithmic time and is implemented as a binary search. 

Cracker Column is a data structure that cannot exist independently in the database, it can only be coupled with Simple Column that supports cracking, They are initialized and attached to a Simple Column instance after the first query. Cracker Columns contain same values as their corresponding Simple Columns, but in a different and constantly changing order. Cracker Columns store tuples in a partially sorted list, that is, once its tuples are reorganized and partitioned into two sublists (one with all tuples whose values are less than or equal to the partition value and one with all tuples whose values are greater than the partition value) Each Cracker Column is supplemented with a  Cracker Index instance. 

The Cracker Index instances are the data structures that are necessary to keep most-up-to date information about all partitions of the Cracker Column tuples. Particularly, Cracker Index stores pairs in the form $(v, p)$, where $v$ indicates that all tuples located at the positions less than and including $p$ have values less that and including $v$.

\subsection{Query processing}
MiniDB queries are the Range Scan objects that operate on a single column. Each Range Scan instance stores the pointer to its column of interest, endpoints of the value ranges, and the range sign, either one of $\leq$, $\textless$, $\geq$, $\textgreater$, $\textless \textless$, $\leq \textless$, $\textless \leq$, $\leq \leq$. Range Scan objects are essentially iterators on the values of their columns of interests. If the column does not support cracking, the iterator goes over all tuples and returns only those whose values belong to the specified range. Otherwise, they use cracking and simply iterate over all values that lie in a specified partition. 


\subsection{Cracker Index implementation}
Cracker Index functionality requires methods such as insert value-position pair, find predecessor of value $v$, find successor of value $v$ and lookup position $p$ of value $v$. Predecessor/successor search is used to define partitions with the closest boundaries, which is the case when a new unseen query arrives. We have implemented Cracker Index using three different underlying representations, AVL Tree, Sorted List and HashMap. Each of them has advantages and disadvantages in different scenarios.

\textbf{AVL Tree} stores value $v$ as a node key and a position $p$ as a node data. All of the operations on AVL Tree have logarithmic cost, which makes it a good candidates for large workloads. However, at the same time, the size of the tree grows with a number of queries, and maintaining the balance of the large tree might be costly.

\textbf{Hash Map} directly stores mapping between the value $v$ and the position $p$. The HashMap implementation is beneficial when all incoming queries are repetitive,and additional cracking is unnecessary, since the correct partition already exists. However, adding a new partition info into the index is costly, as predecessor/successor search takes linear time.

\textbf{Sorted List} stores value-position pairs a list sorted on values. Preserving the order of the list is costly and takes linear time, however, cheap successor and predecessor search balances the query cost.

\begin{table}
\begin{tabular}{| m{2em}  | m{2cm} | m{2cm}|  m{2cm} |} 
\hline
 & \textbf{AVL Tree}& \textbf{Hash Map}&\textbf{Sorted List} \\ 
\hline
\textbf{Pros} & Logarithmic insert, lookup, predecessor and successor search & Constant time lookup & Logarithmic lookup and constant predecessor and successor search \\
\hline
\textbf{Cons} & Cost for balance maintenance & Linear predecessor and successor search & Linear insert \\
\hline
\end{tabular}
\caption{Summary of cost-related advantages and disadvantages for each Cracker Index implementation.}
\label{table:implementations}
\end{table}

In Table \ref{table:implementations} we summarized pros and cons for each Cracker Index implementation.
