\section{System Overview}
In  order to analyze the performance effects of the database cracking, we have implemented our own small database system MiniDB. In this section we will describe components and implementation details of MiniDB.
\label{sec:system}

\subsection{MiniDB overview}
MiniDB is a simple single column database implemented in Java, where each table consists of a single uniquely named column. Database keeps a hash map with (column name, column object) pairs. The database tuples are 32-bit positive integers. Column tuples are stored as a list. MiniDB has three column types: Simple Column, Sorted Column, and Cracker Column. Simple Column does not maintain a specific tuple order and inserts a tuple to the end of the tuple list. In Simple Columns tuple lookup requires linear scan of the tuples list. Sorted Column preserves the order in the tuples list. Tuple lookup takes logarithmic time and is implemented as a binary search. Cracker Columns store tuples in a partially sorted manner, that is, once its tuples are reorganized and partitioned into two sublists (one with all tuples whose values are less than or equal to the partition value and one with all tuples whose values are greater than the partition value).
\subsection{Cracker Index implementation}
