\section{System Overview}
In order to analyze how the query execution time changes due to introducing database cracking techniques, we have implemented our own simple database system MiniDB. In this section we will describe components and implementation details of MiniDB.

MiniDB is a single-column database implemented in Java, where each table consists of a single uniquely named column. The database manager uses a hash map that keeps track of the mappings between columns and their names.  The tables store lists of tuples, where each tuple is a 32-bit integer. MiniDB maintains the following data structures: Simple Column, Sorted Column, Cracker Column, Cracker Index, Range Scan. 
\label{sec:system}

\subsection{MiniDB columns and indexes}
\textbf{Simple Column} instances represent the database tables and store tuples. They do not maintain a specific tuple order and insert a tuple at the end of the tuple list. In Simple Columns tuple lookup requires a linear scan of the entire tuples list. 

\textbf{Sorted Column} is another data structure that represents database tables that preserve the initial order of the tuples. An instance of Sorted Column sorts tuples when it is queried for the first time, using quick sort algorithm. Tuple lookup takes logarithmic time and is implemented as a binary search. 

\textbf{Cracker Column} is a data structure that cannot exist independently in the database, since it can only be coupled with a Simple Column instance that supports cracking. A Cracker Column instance is initialized and attached to a Simple Column instance when the latter is queried for the first time. Cracker Columns contain the same values as their corresponding Simple Columns, but in a different and continuously changing order. Cracker Columns store tuples in a partially sorted list, which means that its tuples are reorganized and partitioned into two sublists, one with all tuples whose values are less than or equal to the partition value, and one with all tuples whose values are greater than the partition value. Each Cracker Column instance is supplemented with a  Cracker Index instance. 

\textbf{Cracker Index} is a data structure that is necessary to keep the most-up-to date information about all partitions of the Cracker Column tuples. In particular, a Cracker Index instance stores pairs in the form $(v, p)$, where $v$ indicates that all tuples located at the positions less than and including $p$ have values less than or equal to $v$.

\subsection{Query processing}
MiniDB queries are represented by the Range Scan objects that operate on a single column each. A Range Scan instance stores the pointer to its column of interest, endpoints of the value ranges, and the range sign, any one of $\leq$, $\textless$, $\geq$, $\textgreater$, $\textless \textless$, $\leq \textless$, $\textless \leq$, $\leq \leq$. Range Scan objects are essentially iterators on the values of their columns of interest. If the column does not support cracking, the iterator goes over all tuples and returns only those whose values belong to the specified range. Otherwise, they use cracking and simply iterate over all values that lie within a specified partition. 


\subsection{Cracker Index implementation}
Cracker Index functionality requires methods such as insert value-position pair, find predecessor of value $v$, find successor of value $v$, and lookup position $p$ of value $v$. Predecessor/successor search is used to define partitions with the closest boundaries, which is the case when a new unseen query arrives. We have implemented Cracker Index using three different underlying representations, AVL Tree, Sorted List, and Hash Map. Each has advantages and disadvantages in different scenarios.

\textbf{AVL Tree} stores value a $v$ as a node key and a position $p$ as a node data. All of the operations on AVL Tree have logarithmic cost, which makes it a good candidate for large workloads. However, at the same time, the size of the tree grows with the number of queries, and maintaining the balance of the large tree might be costly.

\textbf{Hash Map} directly stores mapping between the value $v$ and the position $p$. The Hash Map implementation is beneficial when all incoming queries are repetitive,and additional cracking is unnecessary, since the correct partition already exists. However, adding new partition information into the index is costly, as predecessor/successor search takes linear time.

\textbf{Sorted List} stores value-position pairs in a list sorted by values. Preserving the order of the list is costly and takes linear time; however, cheap successor and predecessor search balances the query cost.

\begin{table}
\begin{tabular}{| m{2em}  | m{2cm} | m{2cm}|  m{2cm} |} 
\hline
 & \textbf{AVL Tree}& \textbf{Hash Map}&\textbf{Sorted List} \\ 
\hline
\textbf{Pros} & Logarithmic insert, lookup, predecessor and successor search & Constant time lookup & Logarithmic lookup and constant predecessor and successor search \\
\hline
\textbf{Cons} & Cost for balance maintenance & Linear predecessor and successor search & Linear insert \\
\hline
\end{tabular}
\caption{Summary of cost-related advantages and disadvantages for each Cracker Index implementation.}
\label{table:implementations}
\end{table}

In Table \ref{table:implementations} we summarize pros and cons for each Cracker Index implementation.
