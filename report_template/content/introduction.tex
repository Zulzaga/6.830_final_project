\section{Introduction}
�Database Cracking� is a concept of partitioning and sorting the data on the fly based on the incoming query workloads. While traditional index creation and maintenance requires upfront knowledge of query patterns and workloads, as well as costly human supervision, database cracking treats index maintenance as a part of query processing. It continuously partitions the database into manageable pieces every time a new query is processed. Dynamic reorganization of data as the system is queried causes subsequent queries to run faster, irrespective of whether these queries had already been run. In our project, we analyzed the relative performance of cracking in a simple single-column database. We conducted multiple experiments, varying implementation details, workloads and
query patterns. We are motivated by the �Database Cracking� paper from Ideros and will have advantages such as no need for upfront knowledge of query patterns and human supervision for index maintenance.
\label{sec:introduction}
