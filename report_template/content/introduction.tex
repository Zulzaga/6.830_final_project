\section{Introduction}

Database cracking is the concept of partitioning and sorting the data on the fly based on the incoming query workload. While traditional index creation and maintenance requires upfront knowledge of query patterns and workloads, as well as costly human supervision, database cracking removes the need for human administration and efficiently manages database workload environments by treating index maintenance as a part of query processing. It continuously partitions the database into manageable pieces every time a new query is processed. Dynamic reorganization of data as the system is queried causes subsequent queries to run faster, irrespective of whether these queries had already been run.

In this paper, we analyze the relative performance of cracking in a simple single-column database that we have implemented for the purposes of this project. We conducted multiple experiments, with varying implementation details, workloads and query patterns. 

Section \ref{sec:background} presents the background on database cracking and related works. Section \ref{sec:system} includes the system overview and implementation details regarding our simplified database. We describe the details of our experiment setup in Section \ref{sec:experiments} and discuss the results of our findings and comparisons to already existing strategies in Section \ref{sec:discussion}. Finally, Section \ref{sec:conclusion} presents our conclusions.

\label{sec:introduction}
