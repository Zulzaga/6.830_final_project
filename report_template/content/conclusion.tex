\section{Conclusion}
Consistent with findings in previous works \cite{idreos_2007, schuhknecht_2014}, we found that as more and more queries arrive, the cost of physical data reorganization is amortized over all the queries, and the cracking approach outperforms both sorted-column scanning and traditional index (simple scanning) under varying workloads. In our simplified environment, we found each of the different cracker index implementations to have certain benefits and drawbacks with respect to each other, but they all outperform the baseline cumulative execution times. Even though MiniDB lacks most of the key features of existing real-world databases, we were still able to verify that the cracking approach is indeed more efficient for queries of the form ``SELECT value  FROM column WHERE value  $\textless$ $x$". 

Overall, by using the cracker index, the database automatically adapts to any workload and creates a re-organization that helps execute subsequent queries faster and faster. We believe that the database cracking idea has great potential in revolutionizing the traditional approach which is database indexes.
\label{sec:conclusion}

