\section{Discussion}
\indent The major concern regarding the results of our experiments and the conclusions we reached is the simplicity of MiniDB. We have implemented our database to be a set of integer arrays; such an implementation lacks many features and components that are present in existing commercial and research databases. For example, we have not used the notions of disks and caches, and our implementation is not cache-conscious.  Furthermore, our database is implemented in Java, which might not be the optimal choice of language for performance-sensitive systems. Additionally, MiniDB is limited to very primitive data structures and does not support more advanced and efficient algorithms. For example, \cite{schuhknecht_2014} mentions more flexible and cost-efficient cracking algorithms, such as hybrid cracking, sideways cracking, and stochastic cracking.

Nevertheless, the concept of database cracking is efficient enough to show positive performance gains even in our simplified environment, thereby proving that it can potentially become an alternative to traditional database indexes. The database cracking idea was shown to be easily repeatable and extendable with varying parameters. The key advantages of the cracking approach is the self-adapting behavior of the database when exposed to workload changes and the consequent elimination of human administration.
\label{sec:discussion}
