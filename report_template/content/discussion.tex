\section{Discussion}
\indent The major concern regarding the results of our experiments and the conclusions we reached is the simplicity of our database-like environment. We have implemented our database to be a set of integer arrays; such implementation is missing many features and components of the existing commercial and research databases. For example, we have not used the notions of disk and cache, and our implementation is not cache-conscious.  Furthermore, our database is implemented in Java, which might not be the optimal choice for performance-sensitive systems. Additionally, MiniDB is limited to very primitive data structures and does not support more advanced and efficient algorithms. For example, \cite{schuhknecht_2014} mention more flexible and cost-efficient cracking algorithms, such as hybrid cracking, sideways cracking, and stochastic cracking.

Nevertheless, the concept of database cracking is efficient enough to show positive performance changes even in our simplified environment, thereby proving that it can potentially become an alternative to traditional database indexes. The database cracking idea was shown to be easily repeatable and extendable with varying parameters. The key advantages of the cracking approach are the self-adapting database based on the workload changes and the consequent elimination of human administration.
\label{sec:discussion}
